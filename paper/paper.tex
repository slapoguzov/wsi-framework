%! Author = Aleksandr Slapoguzov
%! Date = 20.09.2020

% Preamble
\documentclass[11pt]{article}

% Packages
\usepackage{amsmath}
\usepackage[utf8]{inputenc}
\usepackage[russian]{babel}
\usepackage[T1]{fontenc}
\usepackage{multicol}
\usepackage[left=2cm,right=3cm,
    top=2cm,bottom=2cm,bindingoffset=0cm]{geometry}
\usepackage{xcolor}
\usepackage{todonotes}

\title{Разрешение лексической многозначности с помощью кластеризации векторных представлений слов}
\author{Aleksandr Slapoguzov}
% Document
\begin{document}
    \maketitle
    \begin{abstract}
        \textbf{Предмет исследования.} В рамках данной статьи был рассмотрен один из подходов к разрешению
        лексической многозначности в естественных языках.
        Разрешение лексической многозначности - это одна из актуальных задач в области обработки естественных текстов,
        так как является краеугольной проблемой многих других задач в данной области.
        \textbf{Метод.} Для разрешения лексической многозначности использовалась
        кластеризация векторных представлений слов.
        Перевод слов в векторное представление осуществлялся с помощью языковой модели BERT, а для кластеризации
        использовались различные алгоритмы.
        \textbf{Основные результаты.} В ходе экспериментов, проведенных в рамках данной работы,
        алгоритм кластеризации XX\ показал наилучшую точность, которая равна XX.X\%.
        Данный результат сопоставим с другими методами, применяемыми для разрешения лексической многозначности.
        \textbf{Практическая значимость.} Результаты работы могут быть использованы в таких задачах, как информационный
        поиск, извлечение информаций, а также в различных задачах, связанных с семантическими сетями.
    \end{abstract}
    \begin{multicols}{2}
        \section*{Введение}\label{sec:intro}
        С увеличением количества генерируемой информации задача обработки текстов на естественном языке становится всё
        актуальнее.
        Одна из отличительных особенностей таких текстов — это наличие в них неоднозначностей, когда слова
        или некоторые языковые конструкции могут быть интерпретированы по-разному.
        Несмотря на то, что грамматическая и синтаксическая неоднозначность успешно решаются с помощью
        частеречной разметки и анализа зависимостей, морфологической и синтаксической информации недостаточно, чтобы
        однозначно определить значение слова в употребляемом контексте.
        Поэтому возникло такое направление в обработке естественных текстов как \textit{Word Sense Disambiguation(WSD)}
        или \textit{Разрешение лексической многозначности}, задача которого заключается в выборе смысла или значения
        многозначного слова в определенном контексте.
        Обычно выделяют три подхода к WSD: основанный на базах знаний, основанный на обучение с учителем и
        основанный на обучении без учителя.
        Последний также называют \textit{Word Sense Induction(WSI)} и его отличие
        заключается в том, что определяется не конкретное значение слова, а контексты, в котором употребляется слова,
        разбиваются на кластеры, в каждом из которых заданное слово употребляется в одном и том же смысле.
        Рассмотрим следующие примеры со словом "бор":
        \begin{enumerate}
        \item Мы оказались в сосновом \textbf{бору}.
        \item Дужка изготовлена из сплава стали и \textbf{бора}.
        \item Среднее содержание \textbf{бора} в земной коре составляет 4 г/т.
        \end{enumerate}
        Результатом работы WSI будет кластеризация предложения 2 и 3 в одну группу, а предложения 1 в другую.
        \textcolor{red}{ Это не так работает, нужно переписать название статьи:
        Если у анализируемого слова заранее известны возможные смыслы с примерами употребления,
        то задачу разрешения лексической многозначности можно свести к кластеризации анализируемых предложений по
        группам известных смыслов и воспользоваться WSI.}
        \section*{Существующие подходы}\label{sec:related_works}
        Задача WSI исследовалась для русского языка в рамках RUSSE'18[]

    \end{multicols}
\end{document}